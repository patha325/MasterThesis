The \abbrLHC at \abbrCERN is now undergoing an set of upgrades to increase the center of mass energy for the colliding particles to be able to explore new new physical processes. The focus of this thesis lies on the so called phase II upgrade which will preliminarily be completed in 2023. After this upgrade the \abbrLHC will be able to accelerate proton beams to such velocity that each proton has a center of mass energy of 14 TeV.

One disadvantage of this is that it will be harder for the \abbrATLAS detector to isolate unique particle collisions since more and more collisions will occur simultaneously, so called pile-up. 

This thesis focuses on how a mono-jet analysis looking for different \abbrWIMP models of dark matter will be affected by this increase in pile-up rate.

The models which are in focus are effective operators which try to explain dark matter without adding new theories to the standard model or \abbrQFT , such as the D5 operator model and light vector mediator models.

The limits set for the D5 operators mass suppression scale at 14 TeV and 1000 fb$^{-1}$ are 2-3 times better than previous results at 8 TeV and 10 fb$^{-1}$. 

For the first time limits have been set on which mediator masses can be excluded for vector mediator models at 14 TeV.