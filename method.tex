\chapter{Method}\label{cha:meth}
In this chapter the methodology used to perform the tasks given above in \sectionref{sec:goals}.
\newpage
\section{Validation of smearing functions}
Find more information in my presentation. also mention no pile-up dependence of leptons.
For the proposed upgrade of the LHC.

\subsection{Smearing}
One might assume that using a Monte Carlo simulation it would be easy to model and simulate the whole process, from collision to detection and reconstruction in the upgraded LHC. It is possible, but it requires a lot of computing power. Instead one can use one simulation and a mathematical model to calculate the estimated response in the detector. This was validated and used in this thesis to be able to create the data needed for further analysis. 

This was done by using a Monte Carlo simulation of a proton-proton collision, then applying code, that was developed using previous studies \citep{ATL-PHYS-PUB-2013-004} , to simulate the effect that pile-up would have on the signals that come from the detectors and the reconstruction of these. \textbf{Code from where?}

The code uses the experimental data from the previous studies to smear the reconstructed energy and momenta; It does not however alter the direction of the momenta. Other experimental data was used and shows that only jets and $E^{miss}_T$ are affected by pile-up. That this is true can be shown from \textbf{figures and references from nonpileupdep.txt presentation!}. The smearing functions should be given!

\subsection{Validation}
To validate the code comparisons were made with \citep{ATL-PHYS-PUB-2013-004}. 




Parametrization used according to the paper \citep{ATL-PHYS-PUB-2013-004}. What results and what did I get/say in my presentation? Use that in results Perhaps even write something better than the original that can be used to explain this again.

Remember for the discussion to mention different types of rms, relative or absolute. and the problem which occurred with this and the papers faults.

\section{Signal to background ratio}
What I am doing now, looking at what signal? What are the different background processes? What and why was the weight used?

Signals should be explained somewhat in the introduction.
 
\subsection{Selection criteria}
What criteria were used and more importantly why? It is quite important that you can explain why this was used.

\subsection{Comparing with published papers} 	
10fb-1 paper, also can perhaps use CERN-PH-EP-2012-210
\subsection{Figures of merit}
what is it? how is it calculated?
\section{Other selection criteria and observables}
\section{Mitigating the effect}