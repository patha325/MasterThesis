\chapter{Method}\label{cha:meth}
In this chapter the methodology used to perform the tasks given above in ~\sectionref{sec:goals}.
\newpage
\section{Monte Carlo simulation, truth data}
What it is in short detail. How it is set-up and what it produces for us in this context. 
Where does it come from? Perhaps quick explanation of different programs?
\section{ROOT}
A wonderful tool for processing data by programming in C++ and so on... Reference to root homepage?
\section{Validation of smearing functions}
Find more information in my presentation. also mention no pile-up dependence of leptons.
For the proposed upgrade of the LHC.


\subsection{Smearing}
By using the simulation given above one might assume that it would be easy to model and simulate the whole process, from collision to detection in the upgraded LHC. It is possible, but it requires a lot of computing power. Instead one can use one simulation and a mathematical model to calculate the estimated response in the detector. This was validated and used in this thesis to be able to create the data needed for further analysis. The programming code used was the \textbf{official something that can be found somewhere}. 
Several histogram are given regarding this here. Some minor detail about them though most of it should be given in results and discussion also refer perhaps to my presentation? Does one self refer?


Why did we use this, what does it produce for us? A "simple" simulated effect of what can be detected in the detectors without having to run a full simulation. 

The functions smear both the energy and the momenta of the four vector. However not the angles. Explain four vector somewhere?

Explain why leptons were not affected by pile-up. Presentation 2?

\subsection{Validation}
Parametrization used according to the paper \citep{ATL-PHYS-PUB-2013-004}. What results and what did I get/say in my presentation? Use that in results Perhaps even write something better than the original that can be used to explain this again.

Remember for the discussion to mention different types of rms, relative or absolute. and the problem which occurred with this and the papers faults.

\section{Signal to background ratio}
What I am doing now, looking at what signal? What are the different background processes? What and why was the weight used?

Signals should be explained somewhat in the introduction.
 
\subsection{Selection criteria}
What criteria were used and more importantly why? It is quite important that you can explain why this was used.

\subsection{Comparing with published papers} 	
10fb-1 paper, also can perhaps use CERN-PH-EP-2012-210
\subsection{Figures of merit}
what is it? how is it calculated?
\section{Other selection criteria and observables}
\section{Mitigating the effect}