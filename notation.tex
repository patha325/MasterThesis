\begin{notation}% Passing the option "old" to the notation environment will redefine the notationtabular environment so that it produces an old style LaTeX tabular instead of a ctable.sty style tabular.
  \centering

  \begin{notationtabular}{Notations}{Notation}{Explanation}
  barn(b) & 1 barn(b)$=10^{-24}$ cm$^2$ \\
  $\oplus$ & $a\oplus b = \sqrt{a^2 + b^2}$, $a\oplus b\oplus c = \sqrt{a^2+b^2+c^2}$ \\
    %$\naturals$ & Mängden av naturliga tal \\
   \end{notationtabular}

  \begin{notationtabular}{Abbreviations}{Abbreviation}{Expansion}
  
\abbrATLAS\index{ATLAS@\abbrATLAS!abbreviation} & A large Toroidal LHC ApparatuS \\
\abbrCERN\index{CERN@\abbrCERN!abbreviation} & Organisation européenne pour la recherche nucléaire\footnotemark \\
\abbrCMS\index{CMS@\abbrCMS!abbreviation} & Compact Muon Solenoid\\
\abbrCR\index{CR@\abbrCMS!abbreviation} & Control Region\\
\abbrLHC\index{LHC@\abbrLHC!abbreviation} & Large Hadron Collider \\
\abbrMC\index{MC@\abbrMC!abbreviation} & Monte Carlo \\
\abbrSM\index{SM@\abbrSM!abbreviation} & the Standard Model of particle physics \\
\abbrSR\index{SR@\abbrSM!abbreviation} & Signal Region \\
\abbrWIMP\index{WIMP@\abbrWIMPS!abbreviation} & Weakly Interacting Massive Particle \\
\abbrWIMPS\index{WIMPS@\abbrWIMPS!abbreviation} & Weakly Interacting Massive ParticleS \\
\abbrQED\index{QED@\abbrQED!abbreviation} & Quantum ElectroDynamics \\
\abbrQFT\index{QFT@\abbrQFT!abbreviation} & Quantum Field Theory \\
\abbrQM\index{QM@\abbrQM!abbreviation} & Quantum Mechanics \\
  \end{notationtabular}
 
 \footnotetext{Originally, Conseil Européen pour la Recherche Nucléaire}
\end{notation}
