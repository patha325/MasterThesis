\chapter{Results and Conclusions}\label{cha:res}
\textbf{Disclaimer: This chapter is not yet complete.}
\section{Validation of smearing functions}
Have some discussion.

Result they appear to work as expected, the reference paper was a bit unclear, I leave my writing as a better reference.

\section{Signal to background ratio}
\subsection{Limit on M*}
\subsection{Limit on mediator mass}

\section{Other selection criteria and observables}
\subsection{Limit on M*}
\subsection{Limit on mediator mass}
\section{Mitigating the effect of the high luminosity}

\section{Recommendations to mitigate the effect of the high luminosity}
Keep to a higher energy region, or signal region.
\section{Suggestions for future research}
With more time, search for new signal regions, the only solution now for the HL is to go up in energy. Since none of the other parameters (eta,phi etc) seem to be altered these can not be used. Is there something that has been overlooked?

Test the effect of pile-up for lower signal regions? See if the effect is as big as predicted. 

Explore other theoretical models for dark matter, other d operators etc. Models that are based on Supersymmetry and not just effective theories.

Create more similar signal regions to be able to compare to the 20fb paper. 


Sätt av ett kort kapitel sist i rapporten till att avrunda och föreslå rikningar för framtida utveckling av arbetet.
