\chapter{Final remarks}\label{cha:res}
In this thesis an introduction is given to phenomena in elementary particle physics, see \chapterref{cha:intro}. After this a validation of smearing functions is done to validate that the measured resolution is compatible with what is expected, see \chapterref{cha:vali}. In this chapter, \tableref{tab:expected sigma} contains the expected resolutions. 
Finally in \chapterref{cha:darkmatter} two different dark matter signal models are investigated, both an effective theory and a vector mediator model.

Suggestions for future research should focus on that which would have been done if time was not an issue. With an unlimited amount of time \chapterref{cha:darkmatter} would have been expanded to contain several different effective theory models and consider models which are based on Supersymmetry. Another interesting continuation would be to investigate new signal regions to see if it is possible to mitigate the effects of high luminosity even at low energies. This may be possible by using other selection criteria. Similarly it would be interesting to see if there are better selection criteria to increase the ratio of signal to background.  

The most interesting continuation of this work would be to compare the results given here to measurements done after the phase II upgrade and hopefully see characteristic signatures from \abbrWIMPS .